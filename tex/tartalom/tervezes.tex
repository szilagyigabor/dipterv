\chapter{Tervezés}
Először bevezetek néhány jelölést a különböző vektortekere és a közöttük kapcsolatot teremtő operátorokra, amelyek a vizsgált szimulációk közben használatosak. Ehhez az Önálló laboratórium 2 tárgyban vizsgált problémát használom példaként. Itt egy tranziens szimuláción dolgoztam, amivel egy végtelen hosszú, hengeres vezetőben adott időfüggvény szerint változó $I(t)$ össz-árammal adott a gerjesztés és a sugár szerint különböző helyeken felvett $\varphi$ irányú mágneses térerősség, $H(r,t)$ a rendszer válasza. Ez a $H$ eloszlás egyértelműen meghatározza a vezető belsejében az áramsűrűség-eloszlást. A Maxwell-egyenletek átrendezésével adódnak a következő összefüggések az aktuális $H$ térerősség és annak idő szerinti parciális deriváltja között:
\begin{align}
\label{equ:rotrotH}
	\dfrac{\partial H}{\partial t} &= -\dfrac{1}{\mu\sigma}\rot\rot H\\
\label{equ:r=0perem}
	H(r=0,t) &\equiv 0\\
\label{equ:r=Rperem}
	H(r=R,t) &= 2R\pi I(t)
\end{align}
ahol $\mu$ és $\sigma$ rendre a vezető mágneses permeabilitása és fajlagos vezetőképessége, $R$ pedig a vezető sugara. A feladatban az $r$ sugár szerinti $(0,R)$ intervallum az $\Omega$ vizsgált tartomány, ennek a két végpontja, $\{0,R\}$ a $\Gamma$ perem, \aref{equ:r=0perem} és \aref{equ:r=Rperem} a két peremfeltétel. \Aref{equ:rotrotH}. egyenletben szereplő $-1/(\mu\sigma)\cdot\rot\rot(.)$ operátor, amit most $L$-lel jelölök, két függvénytér közötti lineáris leképezés, vagyis $L: F \rightarrow G$. Az értelmezési tartománya a feladat keretein belül egy olyan $F: [0,R] \rightarrow \mathbb{R}$ függvénytér, ami a lehetséges $H(r,t=\tau)$ függvények halmaza. Itt $\tau$ helyére mindig valamilyen konkrét időpillanatot kell behelyettesíteni, így kapunk egy csak $r$-től függő függvényt. $L$ értékkészlete pedig egy $G: (0,R) \rightarrow \mathbb{R}$ függvénytér, ami pedig a lehetséges $\partial H/\partial t$ függvények halmaza. Azért csak $(0,R) = [0,R]\setminus\{0,R\} = \Omega$ $G$ értelmezési tartománya, mert a peremeken már a peremfeltételek miatt explicite adott $H$ értéke, így azt ott nem kell meghatározni. Ez a különbség akkor válik fontossá, amikor véges dimenziós, vagyis véges szabadsági fokú terekkel közelítjük $F$-et és $G$-t, például az időtartománybeli véges differenciák módszerével (FDTD-vel) vagy végeselem-módszerrel (FEM-mel). Ilyen számítógépes módszereknél egy ,,szabadsági fok'' egy számértéket jelent, ami lehet páldául egy pontban felvett térerősségérték vagy egy adott szakaszon annak a vonalmenti integrálja.

Eljutottam tehát a jelölések bevezetéséhez. Azt, hogy egy operátor, mint a fenti $L$ operátor értelmezési tartománya és értékkészlete milyen fizikai jelentéssel bír és ezek hol értelmezettek, a következőképpen jelölöm. $L: H_{\Omega, \Gamma} \rightarrow H_{\Omega}$ azt jelenti, hogy ezt az operátort arra használatos, hogy az $\Omega$ tartomány belsejében és a $\Gamma$ peremen felvett $H$ mágneses térerősségértékek alapján kiszámoljuk $\Omega$ belsejében felvett térerősségeket, vagy azok időbeli parciális deriváltjait tehát a $\Gamma$ peremen nem kapunk ezekre eredményt. Erre azt lehet mondani, hogy $L$ egy ,,téglalap alakú'' operátor, mivel a bemenete és kimenete különböző számú szabadsági fokkal rendelkezik, kicsit gyakorlatiasabban különböző hosszúságú vektorok. Ellenkező esetben az operátor ,,négyzet alakú''.

Ez így nem igaz, a rotrot operátor pont, hogy négyzet alakú. Csak érvényesíteni kell a peremfeltételeket is, de valójában a bemenete csak a $H_{\Omega}$.

Az Önálló laboratórium 2 tárgyban vizsgált egydimenziós feladatnál meglehetősen egyszerű volt az időléptetéshez felírni a POD-val redukált rendű alakot. Itt ugyanis a $H$ tér változását ($\partial H/\partial t$) csak az adott időlépésben érvényes Dirichlet-peremfeltétel és maga a $H$ határozta meg. A $\partial H/\partial t$ kiszámítása a fenti bemeneti adatok alapján egy mátrix-szorzásként adódik. Így elég csak a peremfeltételeket nem tartalmazó $H$ vektorok alapján számolni egy POD redukált bázist $H$-ra, majd a mátrix alakban adott $H \rightarrow H$